\documentclass{article}

\author{Stephen M. Lee}
\title{A NLP Approach to Understanding Variation in Political News}

\begin{document}
	\maketitle 
	
	\section{Introduction}
	The 2016 United States Presidential Election, to many, raised questions as to the reliability of their news sources [SOURCE NEEDED]. Since then, many social media companies and other institutions have begun public campaigns to combat the perceived threat from fake news, with arguably limited results [SOURCE NEEDED]. With a goal of news article source identification, I scrape several thousand news articles from Fox, Vox, and PBS. By some estimations [SOURCE NEEDED], these three represent distinct categories of news: Fox is often considered extreme “right” opinion (i.e. conservative); Vox is considered extreme “left” opinion (i.e. liberal); and PBS is considered “center” primary source news. I first calculate the top word, 2-gram, and 3-gram frequencies to better understand the dataset, and then train a Bidirectional LSTM neural network with pretrained embeddings to classify the news source. 
	
	\section{Data}
	
	\section{Predictive Models}
	
	\section{Analysis}
	
	\section{Conclusion and Discussion}
	
\end{document}